\documentclass{article}
\usepackage{graphicx} % Required for inserting images
\usepackage{hyperref}

\title{Catalan Programming Assignment: \\Written Report}
\author{Your Name}

\begin{document}

\maketitle

\section{Introduction}
Introduce the problem you are solving here. Assume the audience has no background knowledge of Catalan.

\subsection{Introduction to Latex -- delete this when submitting}
Latex is a programming-inspired text editor. Write your text using Latex commands and upon compilation, Latex will automatically format it prettily for you. Here are some examples and guidelines of things you can do in Latex.

In order to begin a new paragraph, you must have two new lines. You can also use \\ to force a new line. If you begin a new paragraph normally, it will automatically be indented for you.

You can style mathematical equations $x+2=y$ directly in-line. You can \textbf{bold} and \textit{italicize} words, and write them in \texttt{code font}. Latex uses separate apostrophes for left- and right-facing apostrophes. The regular tick ' creates a normal apostrophe, while the backtick ` will create an opposite-facing apostrophe. You can add links: \url{https://www.baeldung.com/java-generics}.

You can create a table using the `tabular' environment.

\begin{center}
\begin{tabular}{|c|c|} 
 \hline
 Header A & Header B \\
 \hline
 \hline
 A1 & B1 \\ 
 \hline
 A2 & B2 \\ 
 \hline
 A3 & B3 \\ 
 \hline
 A4 & B4 \\ 
 \hline
\end{tabular}
\end{center}

You can write lists using the `itemize' environment.

\begin{itemize}
    \item Here is a list point
    \item Here is another one
\end{itemize}

\section{Overview of Classes}

\subsection{Vertex}
Introduce and discuss your implementation of the Vertex class.

\subsection{Graph}
Introduce and discuss your implementation of the Graph class.

\subsection{Move}
Introduce and discuss your implementation of the Move class.

\subsection{Catalan}
Introduce and discuss your implementation of the Catalan class.

\subsubsection{Solution}
Discuss in detail how you solved the game of Catalan (ie. the \texttt{solve()} method).

\section{Programming Principles}
Discuss which programming principles you used, where you used them, and why.

\section{Additional Functionality}
You can delete this section if you did not add any additional functionality.

\section{Challenges}
Discuss the challenges and/or bugs you faced and how you overcame them.

\end{document}

